\documentclass[12pt,a4paper]{article}
\usepackage[utf8]{inputenc}
\usepackage{cmap}
\usepackage[OT1,T2A]{fontenc}
\usepackage[russian]{babel}
\usepackage{graphicx}
\usepackage[margin=15mm]{geometry}
\usepackage{amssymb}
\usepackage{tikz}
\usepackage{paralist}

% основные библиотеки
\usepackage{pgfplots, tikz, circuitikz}
\usetikzlibrary{%
arrows,%
calc,%
patterns,%
intersections,%
decorations.pathreplacing,%
decorations.pathmorphing,%
decorations.text,%
decorations.markings,%
shapes,
}
% основные используемые стили
% стиль для стрелки
\tikzset{>=latex,%
every path/.style=thick,%
% платформа: пол или потолок
interface/.style={postaction={draw,decorate,decoration={border,
angle=45, amplitude=0.2cm, segment length=1mm}}},%
% пружина
spring/.style={decorate,decoration={coil,amplitude=1mm, segment
length=1mm},thick},%
% заряд, вершина, просто точка
dot/.style={inner sep=0mm,minimum size=0.1cm,fill,circle},%
% стрелка в середине отрезка
marrow/.style={postaction={draw,decorate,decoration={markings,
mark=at position 0.6 with {\arrow{latex}}}}}}
% обозначение угла
\tikzset{arcnode/.style={
decoration={
markings, raise = 2mm,
mark=at position 0.5 with {
\node[inner sep=0] {#1};
}
},
postaction={decorate}
}
}
% команда для отметки угла: проводит дугу между двумя лучами,
% проведёнными между точками #2--#3 и #2--#4
% первый аргумент - необязательный, стиль линии
% #2,#3,#4 - точки
% #5 - радиус дуги для обозначения угла
% #6 - обозначение угла, например, $\alpha$
\newcommand*\marktheangle[6][]{
\draw[thick,arcnode={#6},#1] let \p2=($(#3)-(#2)$),%
\p3=($(#4)-(#2)$),%
\n2 = {atan2(\x2,\y2)},%
\n3 = {atan2(\x3,\y3)}%
in ($(\n2:#5)+(#2)$) arc (\n2:\n3:#5);
}
% для работы с графиками
\pgfplotsset{compat=newest}
% библиотеки для электричества
\usetikzlibrary{circuits.ee,circuits.ee.IEC}
% амперметр
\tikzset{circuit declare symbol = ammeter}
\tikzset{set ammeter graphic ={draw,generic circle IEC, minimum
size=5mm,info=center:A}}
% вольтметр
\tikzset{circuit declare symbol = voltmeter}
\tikzset{set voltmeter graphic ={draw,generic circle IEC, minimum
size=5mm,info=center:V}}
% кружок
\tikzset{circuit declare symbol = meter}
\tikzset{set meter graphic ={draw,generic circle IEC, minimum
size=5mm}} 
%необходимые пакеты
\usepackage{tabularx}
\usepackage{makecell}
% \task{НОМЕР ЗАДАЧИ}{УСЛОВИЕ ЗАДАЧИ}
% задача без картинки
% оформлена как таблица с двумя колонками
% ширина первой колонки (номер столбца) фиксирована, 0.3cm
% ширина второй колонки автоматически рассчитывается из ширины
% страницы (с учётом всевозможных отступов)
\newcommand{\task}[2]{
\begin{tabularx}{\textwidth}{|c|X|}
\cline{1-2}
\makecell*[{{p{0.5cm}}}]{ \centering #1 } &
\makecell*[{{p{\hsize}}}]{ #2 } \\
\cline{1-2}
\end{tabularx}

\vspace{-1pt}

}
% \taskpic[ШИРИНА КАРТИНКИ]{НОМЕР ЗАДАЧИ}{УСЛОВИЕ ЗАДАЧИ}{КАРТИНКА}
% задача с картинкой
% оформлена как таблица с тремя колонками
% первый аргумент - необязательный, по умолчанию ширина картинки равна
% 4cm, но можно выставить свою
% ширина второй колонки (условие задачи) рассчитывается из ширины
% страницы и ширины картинки
\newcommand{\taskpic}[4][4cm]{
\begin{tabularx}{\textwidth}{|c|X|c|}
\cline{1-3}
\makecell*[{{p{0.5cm}}}]{ \centering #2 } &
\makecell*[{{p{\hsize}}}]{ #3 } &
\makecell*[{{p{#1}}}]{ \centering #4} \\
\cline{1-3}
\end{tabularx}

\vspace{-1pt}

}

\parindent=0cm


\pagestyle{empty}
\graphicspath{ {images/} }


\begin{document}
\begin{center}
\begin{Large}
\textsc{ГЦФО. 9 класс. 2014/15.}
\end{Large}
\end{center}
\taskpic{5}{В солдата, сидящего в окопе, неприятель выстрелил из мортиры (см. рис.). Снаряд летел ровно на него, но до окопа не долетел. С точки зрения солдата снаряд поднимался в течение $t_1$ секунд, а опускался быстрее, за $t_2$ секунд, смотрел он из окопа от уровня земли. Известно, что неприятельские мортиры стреляют под углом $\alpha$ к горизонту, а модуль начальной скорости снаряда равен $V_0$. Найдите, на каком расстоянии от окопа упал снаряд. Сопротивлением воздуха пренебречь, ускорение свободного падения равно $g$.}{\includegraphics[width=4cm]{5}}
\taskpic{6}{Наклонная плоскость образует угол $\alpha$ с горизонтом. С высоты $H$ на нее падает мячик. Считая удары мячика о плоскость абсолютно упругими, определите расстояние между точками $n$-го и $(n+1)$-го отскока мячика от плоскости.}{\includegraphics[width=3cm]{6}}
\taskpic{7}{Тело соскальзывает с гладкой горки с высоты $H$. Отрыв тела от горки происходит на высоте $h$, при этом скорость тела горизонтальна. При каком значении $h$ дальность полета тела будет максимальной?}{\includegraphics[width=4cm]{7}}
\taskpic{9}{Две прямые, пересекающиеся под углом $\alpha$, движутся перпендикулярно самим себе со скоростями $v_1$ и $v_2$.  Определите скорость $v$ точки пересечения прямых.}{\includegraphics[width=3cm]{9}}
\taskpic{10}{Чиполлино решил сделать святящуюся эмблему своего любимого футбольного клуба <<Зенит>>. Он собрал электрическую схему, как показано на рисунке. Все буквы он составил из неоновых лампочек с одинаковым сопротивлением $R$ (на рисунке толстые черные линии между серыми точками) и соединил их проводами (сопротивление которых пренебрежимо мало, на рисунке тонкие линии). Лампа начинает светиться, если через нее течет сколь угодно малый ток. Нарисуйте, как выглядела светящаяся часть названия клуба, когда Чиполлино подключил напряжение к клеммам 1 и 2. Ответ поясните.}{\includegraphics[width=4cm]{10}}
\task{11}{Два одинаковых вольтметра соединили параллельно, третий вольтметр подключили к этой комбинации последовательно, и к концам получившейся цепи присоединили идеальную батарейку. При этом вольтметры показывают 4 В, 4 В и 5 В. Какое напряжение у батарейки? Могут ли быть одинаковыми все три вольтметра? Что покажут эти же приборы, если их все соединить последовательно и подключить к той же батарейке? Показания приборов считайте точными.}

\end{document}