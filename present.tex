\documentclass[12pt,a4paper]{article}
\usepackage[utf8]{inputenc}
\usepackage{cmap}
\usepackage[OT1,T2A]{fontenc}
\usepackage[russian]{babel}
\usepackage{graphicx}
\usepackage[margin=15mm]{geometry}
\usepackage{amssymb}
\usepackage{tikz}
\usepackage{paralist}

% основные библиотеки
\usepackage{pgfplots, tikz, circuitikz}
\usetikzlibrary{%
arrows,%
calc,%
patterns,%
intersections,%
decorations.pathreplacing,%
decorations.pathmorphing,%
decorations.text,%
decorations.markings,%
shapes,
}
% основные используемые стили
% стиль для стрелки
\tikzset{>=latex,%
every path/.style=thick,%
% платформа: пол или потолок
interface/.style={postaction={draw,decorate,decoration={border,
angle=45, amplitude=0.2cm, segment length=1mm}}},%
% пружина
spring/.style={decorate,decoration={coil,amplitude=1mm, segment
length=1mm},thick},%
% заряд, вершина, просто точка
dot/.style={inner sep=0mm,minimum size=0.1cm,fill,circle},%
% стрелка в середине отрезка
marrow/.style={postaction={draw,decorate,decoration={markings,
mark=at position 0.6 with {\arrow{latex}}}}}}
% обозначение угла
\tikzset{arcnode/.style={
decoration={
markings, raise = 2mm,
mark=at position 0.5 with {
\node[inner sep=0] {#1};
}
},
postaction={decorate}
}
}
% команда для отметки угла: проводит дугу между двумя лучами,
% проведёнными между точками #2--#3 и #2--#4
% первый аргумент - необязательный, стиль линии
% #2,#3,#4 - точки
% #5 - радиус дуги для обозначения угла
% #6 - обозначение угла, например, $\alpha$
\newcommand*\marktheangle[6][]{
\draw[thick,arcnode={#6},#1] let \p2=($(#3)-(#2)$),%
\p3=($(#4)-(#2)$),%
\n2 = {atan2(\x2,\y2)},%
\n3 = {atan2(\x3,\y3)}%
in ($(\n2:#5)+(#2)$) arc (\n2:\n3:#5);
}
% для работы с графиками
\pgfplotsset{compat=newest}
% библиотеки для электричества
\usetikzlibrary{circuits.ee,circuits.ee.IEC}
% амперметр
\tikzset{circuit declare symbol = ammeter}
\tikzset{set ammeter graphic ={draw,generic circle IEC, minimum
size=5mm,info=center:A}}
% вольтметр
\tikzset{circuit declare symbol = voltmeter}
\tikzset{set voltmeter graphic ={draw,generic circle IEC, minimum
size=5mm,info=center:V}}
% кружок
\tikzset{circuit declare symbol = meter}
\tikzset{set meter graphic ={draw,generic circle IEC, minimum
size=5mm}} 
%необходимые пакеты
\usepackage{tabularx}
\usepackage{makecell}
% \task{НОМЕР ЗАДАЧИ}{УСЛОВИЕ ЗАДАЧИ}
% задача без картинки
% оформлена как таблица с двумя колонками
% ширина первой колонки (номер столбца) фиксирована, 0.3cm
% ширина второй колонки автоматически рассчитывается из ширины
% страницы (с учётом всевозможных отступов)
\newcommand{\task}[2]{
\begin{tabularx}{\textwidth}{|c|X|}
\cline{1-2}
\makecell*[{{p{0.5cm}}}]{ \centering #1 } &
\makecell*[{{p{\hsize}}}]{ #2 } \\
\cline{1-2}
\end{tabularx}

\vspace{-1pt}

}
% \taskpic[ШИРИНА КАРТИНКИ]{НОМЕР ЗАДАЧИ}{УСЛОВИЕ ЗАДАЧИ}{КАРТИНКА}
% задача с картинкой
% оформлена как таблица с тремя колонками
% первый аргумент - необязательный, по умолчанию ширина картинки равна
% 4cm, но можно выставить свою
% ширина второй колонки (условие задачи) рассчитывается из ширины
% страницы и ширины картинки
\newcommand{\taskpic}[4][4cm]{
\begin{tabularx}{\textwidth}{|c|X|c|}
\cline{1-3}
\makecell*[{{p{0.5cm}}}]{ \centering #2 } &
\makecell*[{{p{\hsize}}}]{ #3 } &
\makecell*[{{p{#1}}}]{ \centering #4} \\
\cline{1-3}
\end{tabularx}

\vspace{-1pt}

}

\parindent=0cm


\pagestyle{empty}
\graphicspath{ {images/} }


\begin{document}

\begin{center}
\begin{Large}
\textsc{ГЦФО. 9 класс. 2014/15.}
\end{Large}
\end{center}

\footnotesize

\task{41}{Поршень массы $M = 2$~кг может с трением скользить внутри вертикальной неподвижной трубы. Сначала поршень прикрепили внутри трубы к потолку пружиной жесткостью $k_1 = 20$~Н/м, длина которой в нерастянутом состоянии $l_1=60$~см. Поршень расположили на уровне середины трубы, отпустили, и он остался неподвижен. Затем опыт повторили, поменяв пружину - жесткость новой пружины стала $k_2 = 10$~Н/м, а длина в нерастянутом состоянии $l_2 = 20$~см. Удивительно, но поршень в середине трубы снова остался неподвижен. При каких значениях силы трения поршня о трубу это возможно? Влиянием воздуха пренебречь, $g = 10$~м/с$^2$.}
\taskpic{43}{Велосипед с колесами, имеющими форму равностороннего треугольника, за время $t$ прошел по дороге достаточно большое расстояние $s$. Найдите среднее значение модуля скорости точки, расположенной в вершине колеса. Колеса не проскальзывают по дороге, велосипед не отрывается от земли.}{\includegraphics[width=3cm]{43}}
\taskpic{44}{Экспериментатор взял 4 одинаковых металлических стержня и собрал из них Y-образную фигуру. К концам фигуры экспериментатор присоединил 3 одинаковых больших металлических шара, имеющих температуру $t_1 = 0 ^\circ$C, $t_2 = 50 ^\circ$C и $t_3 = 100 ^\circ$C (см. рис.). Экспериментатор обеспечил хороший тепловой контакт стержней с шарами и другими стержнями. Через некоторое время он обнаружил, что первый шар нагрелся на $0{,}4^\circ$C. Какую температуру имели в этот момент два других шара? Считайте, что теплоемкость стержней пренебрежимо мала, а теплообмен с окружающей средой отсутствует. Мощность теплопередачи по стержню пропорциональна разности температур на его концах.}{\includegraphics[width=3cm]{44}}
\taskpic{45}{Маленький шарик массы $m$, закрепленный на вертикальной пружине, расположили под столом с отверстием, в положении равновесия шарик находится посередине отверстия. Обнаружилось, что если шарик отклонить вниз на произвольное расстояние и отпустить, он колеблется вокруг положения равновесия с периодом $T_0$. Над отверстием поставили тело массой $m$ (см. рис.) и снова вывели шарик из положения равновесия. Определить период колебаний системы, если известно, что максимальная скорость шарика $v_m$. Шарик и тело соударяются абсолютно упруго; тело, подскакивая, движется строго вертикально. Сопротивлением воздуха пренебречь, ускорение свободного падения~$g$.}{\includegraphics[width=4cm]{45}}
\taskpic{46}{Длинный однородный брусок с поперечным сечением в виде прямоугольника со сторонами $a \neq b$ подвешен на двух вертикальных нитях, прикрепленных к одному из ребер, над сосудом, в который наливают воду. Когда в сосуд налили некоторое количество воды, два ребра бруска оказались точно на поверхности воды (вид сбоку, со стороны вышеупомянутого поперечного сечения, показан на рисунке). Найдите плотность материала, из которого сделан брусок. Плотность воды $\rho = 1$~г/см$^3$.\\
Примечание: центр масс однородного треугольника расположен на пересечении его медиан.}{\includegraphics[width=4cm]{46}}
\taskpic{47}{В <<черном ящике>> находится схема, состоящая из последовательно соединенных идеальной батарейки с напряжением $U_0 = 3{,}3$~В и резистора сопротивлением $R_0 = 1500$~Ом (рисунок слева). При попытке изготовить второй такой же <<черный ящик>> оказалось, что батареек с нужным напряжением 3{,}3~В в лаборатории больше нет, зато есть другая идеальная батарейка с напряжением $U_1 = 5$~В. По этой причине решили собрать схему, состоящую из имеющейся батарейки и двух резисторов, соединив эти элементы так, как изображено на рисунке справа. Найдите, какими должны быть сопротивления резисторов $R_1$ и $R_2$ для того, чтобы два этих <<черных ящика>> оказались эквивалентными друг другу.}{\includegraphics[width=4cm]{47}}
\end{document}
