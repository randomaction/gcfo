\documentclass[12pt,a4paper]{article}
\usepackage[utf8]{inputenc}
\usepackage{cmap}
\usepackage[OT1,T2A]{fontenc}
\usepackage[russian]{babel}
\usepackage{graphicx}
\usepackage[margin=15mm]{geometry}
\usepackage{amssymb}
\usepackage{tikz}
\usepackage{paralist}

% основные библиотеки
\usepackage{pgfplots, tikz, circuitikz}
\usetikzlibrary{%
arrows,%
calc,%
patterns,%
intersections,%
decorations.pathreplacing,%
decorations.pathmorphing,%
decorations.text,%
decorations.markings,%
shapes,
}
% основные используемые стили
% стиль для стрелки
\tikzset{>=latex,%
every path/.style=thick,%
% платформа: пол или потолок
interface/.style={postaction={draw,decorate,decoration={border,
angle=45, amplitude=0.2cm, segment length=1mm}}},%
% пружина
spring/.style={decorate,decoration={coil,amplitude=1mm, segment
length=1mm},thick},%
% заряд, вершина, просто точка
dot/.style={inner sep=0mm,minimum size=0.1cm,fill,circle},%
% стрелка в середине отрезка
marrow/.style={postaction={draw,decorate,decoration={markings,
mark=at position 0.6 with {\arrow{latex}}}}}}
% обозначение угла
\tikzset{arcnode/.style={
decoration={
markings, raise = 2mm,
mark=at position 0.5 with {
\node[inner sep=0] {#1};
}
},
postaction={decorate}
}
}
% команда для отметки угла: проводит дугу между двумя лучами,
% проведёнными между точками #2--#3 и #2--#4
% первый аргумент - необязательный, стиль линии
% #2,#3,#4 - точки
% #5 - радиус дуги для обозначения угла
% #6 - обозначение угла, например, $\alpha$
\newcommand*\marktheangle[6][]{
\draw[thick,arcnode={#6},#1] let \p2=($(#3)-(#2)$),%
\p3=($(#4)-(#2)$),%
\n2 = {atan2(\x2,\y2)},%
\n3 = {atan2(\x3,\y3)}%
in ($(\n2:#5)+(#2)$) arc (\n2:\n3:#5);
}
% для работы с графиками
\pgfplotsset{compat=newest}
% библиотеки для электричества
\usetikzlibrary{circuits.ee,circuits.ee.IEC}
% амперметр
\tikzset{circuit declare symbol = ammeter}
\tikzset{set ammeter graphic ={draw,generic circle IEC, minimum
size=5mm,info=center:A}}
% вольтметр
\tikzset{circuit declare symbol = voltmeter}
\tikzset{set voltmeter graphic ={draw,generic circle IEC, minimum
size=5mm,info=center:V}}
% кружок
\tikzset{circuit declare symbol = meter}
\tikzset{set meter graphic ={draw,generic circle IEC, minimum
size=5mm}} 
%необходимые пакеты
\usepackage{tabularx}
\usepackage{makecell}
% \task{НОМЕР ЗАДАЧИ}{УСЛОВИЕ ЗАДАЧИ}
% задача без картинки
% оформлена как таблица с двумя колонками
% ширина первой колонки (номер столбца) фиксирована, 0.3cm
% ширина второй колонки автоматически рассчитывается из ширины
% страницы (с учётом всевозможных отступов)
\newcommand{\task}[2]{
\begin{tabularx}{\textwidth}{|c|X|}
\cline{1-2}
\makecell*[{{p{0.5cm}}}]{ \centering #1 } &
\makecell*[{{p{\hsize}}}]{ #2 } \\
\cline{1-2}
\end{tabularx}

\vspace{-1pt}

}
% \taskpic[ШИРИНА КАРТИНКИ]{НОМЕР ЗАДАЧИ}{УСЛОВИЕ ЗАДАЧИ}{КАРТИНКА}
% задача с картинкой
% оформлена как таблица с тремя колонками
% первый аргумент - необязательный, по умолчанию ширина картинки равна
% 4cm, но можно выставить свою
% ширина второй колонки (условие задачи) рассчитывается из ширины
% страницы и ширины картинки
\newcommand{\taskpic}[4][4cm]{
\begin{tabularx}{\textwidth}{|c|X|c|}
\cline{1-3}
\makecell*[{{p{0.5cm}}}]{ \centering #2 } &
\makecell*[{{p{\hsize}}}]{ #3 } &
\makecell*[{{p{#1}}}]{ \centering #4} \\
\cline{1-3}
\end{tabularx}

\vspace{-1pt}

}

\parindent=0cm


\pagestyle{empty}
\graphicspath{ {images/} }


\begin{document}

\begin{center}
\begin{Large}
\textsc{ГЦФО. 9 класс. 2014/15.}
\end{Large}
\end{center}

\small

\taskpic[35mm]{44}{Экспериментатор взял 4 одинаковых металлических стержня и собрал из них Y-образную фигуру. К концам фигуры экспериментатор присоединил 3 одинаковых больших металлических шара, имеющих температуру $t_1 = 0 ^\circ$C, $t_2 = 50 ^\circ$C и $t_3 = 100 ^\circ$C (см. рис.). Экспериментатор обеспечил хороший тепловой контакт стержней с шарами и другими стержнями. Через некоторое время он обнаружил, что первый шар нагрелся на $0{,}4^\circ$C. Какую температуру имели в этот момент два других шара? Считайте, что теплоемкость стержней пренебрежимо мала, а теплообмен с окружающей средой отсутствует. Мощность теплопередачи по стержню пропорциональна разности температур на его концах.}{\includegraphics[width=2cm]{44}}
\taskpic[35mm]{45}{Маленький шарик массы $m$, закрепленный на вертикальной пружине, расположили под столом с отверстием, в положении равновесия шарик находится посередине отверстия. Обнаружилось, что если шарик отклонить вниз на произвольное расстояние и отпустить, он колеблется вокруг положения равновесия с периодом $T_0$. Над отверстием поставили тело массой $m$ (см. рис.) и снова вывели шарик из положения равновесия. Определить период колебаний системы, если известно, что максимальная скорость шарика $v_m$. Шарик и тело соударяются абсолютно упруго; тело, подскакивая, движется строго вертикально. Сопротивлением воздуха пренебречь, ускорение свободного падения~$g$.}{\includegraphics[width=3cm]{45}}
\taskpic[35mm]{47}{В <<черном ящике>> находится схема, состоящая из последовательно соединенных идеальной батарейки с напряжением $U_0 = 3{,}3$~В и резистора сопротивлением $R_0 = 1500$~Ом (рисунок слева). При попытке изготовить второй такой же <<черный ящик>> оказалось, что батареек с нужным напряжением 3{,}3~В в лаборатории больше нет, зато есть другая идеальная батарейка с напряжением $U_1 = 5$~В. По этой причине решили собрать схему, состоящую из имеющейся батарейки и двух резисторов, соединив эти элементы так, как изображено на рисунке справа. Найдите, какими должны быть сопротивления резисторов $R_1$ и $R_2$ для того, чтобы два этих <<черных ящика>> оказались эквивалентными друг другу.}{\includegraphics[width=3cm]{47}}
\task{48}{Для изготовления нагревательной спирали кипятильника взяли проволоку длиной $l_1$. После подключения этого кипятильника к источнику напряжения с малым внутренним сопротивлением на нагревание некоторой массы воды в калориметре на $50^\circ$C было затрачено время $\tau_1 = 2$~минуты. Затем проволоку, из которой была сделана спираль кипятильника, расплавили и изготовили из расплава новую проволоку длиной $l_2 = 2l_1$. Из новой проволоки сделали другую спираль для кипятильника, опустили его в другой калориметр с другим количеством воды, и подключили кипятильник к тому же источнику напряжения. На нагревание воды на $50^\circ$C во втором калориметре было потрачено время $\tau_2 = 12$~минут. Во сколько раз масса воды во втором калориметре отличается от массы воды в первом калориметре? Считайте, что потерь теплоты при нагревании воды не происходит, теплоемкости калориметров пренебрежимо малы, а плотность и проводимость металла после переплавки остаются прежними.}
\task{49}{Муха заметила на столе каплю меда, пролетая точно над ней горизонтально со скоростью $v_0$ на высоте $H$. Как надо двигаться мухе, чтобы как можно быстрее добраться до меда? Сколько времени $t$ для этого понадобится? Считайте, что муха способна развивать ускорение $a$ в любом направлении.}
\task{50}{На вертикальной оси электродвигателя укреплен отвес --- маленький шарик на нити длиной $l=12{,}5$~см. При медленном вращении двигателя нить остается вертикальной, а при быстром вращении шарик движется как конический маятник. При какой частоте вращения $n_1$ нить начинает отклоняться от вертикали? Чему равен угол ее отклонения $\varphi_2$ при частоте вращения $n_2=3$~с$^{-1}$?}
\end{document}
