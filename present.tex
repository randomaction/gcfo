\documentclass[12pt,a4paper]{article}
\usepackage[utf8]{inputenc}
\usepackage{cmap}
\usepackage[OT1,T2A]{fontenc}
\usepackage[russian]{babel}
\usepackage{graphicx}
\usepackage[margin=15mm]{geometry}
\usepackage{amssymb}
\usepackage{tikz}
\usepackage{paralist}

% основные библиотеки
\usepackage{pgfplots, tikz, circuitikz}
\usetikzlibrary{%
arrows,%
calc,%
patterns,%
intersections,%
decorations.pathreplacing,%
decorations.pathmorphing,%
decorations.text,%
decorations.markings,%
shapes,
}
% основные используемые стили
% стиль для стрелки
\tikzset{>=latex,%
every path/.style=thick,%
% платформа: пол или потолок
interface/.style={postaction={draw,decorate,decoration={border,
angle=45, amplitude=0.2cm, segment length=1mm}}},%
% пружина
spring/.style={decorate,decoration={coil,amplitude=1mm, segment
length=1mm},thick},%
% заряд, вершина, просто точка
dot/.style={inner sep=0mm,minimum size=0.1cm,fill,circle},%
% стрелка в середине отрезка
marrow/.style={postaction={draw,decorate,decoration={markings,
mark=at position 0.6 with {\arrow{latex}}}}}}
% обозначение угла
\tikzset{arcnode/.style={
decoration={
markings, raise = 2mm,
mark=at position 0.5 with {
\node[inner sep=0] {#1};
}
},
postaction={decorate}
}
}
% команда для отметки угла: проводит дугу между двумя лучами,
% проведёнными между точками #2--#3 и #2--#4
% первый аргумент - необязательный, стиль линии
% #2,#3,#4 - точки
% #5 - радиус дуги для обозначения угла
% #6 - обозначение угла, например, $\alpha$
\newcommand*\marktheangle[6][]{
\draw[thick,arcnode={#6},#1] let \p2=($(#3)-(#2)$),%
\p3=($(#4)-(#2)$),%
\n2 = {atan2(\x2,\y2)},%
\n3 = {atan2(\x3,\y3)}%
in ($(\n2:#5)+(#2)$) arc (\n2:\n3:#5);
}
% для работы с графиками
\pgfplotsset{compat=newest}
% библиотеки для электричества
\usetikzlibrary{circuits.ee,circuits.ee.IEC}
% амперметр
\tikzset{circuit declare symbol = ammeter}
\tikzset{set ammeter graphic ={draw,generic circle IEC, minimum
size=5mm,info=center:A}}
% вольтметр
\tikzset{circuit declare symbol = voltmeter}
\tikzset{set voltmeter graphic ={draw,generic circle IEC, minimum
size=5mm,info=center:V}}
% кружок
\tikzset{circuit declare symbol = meter}
\tikzset{set meter graphic ={draw,generic circle IEC, minimum
size=5mm}} 
%необходимые пакеты
\usepackage{tabularx}
\usepackage{makecell}
% \task{НОМЕР ЗАДАЧИ}{УСЛОВИЕ ЗАДАЧИ}
% задача без картинки
% оформлена как таблица с двумя колонками
% ширина первой колонки (номер столбца) фиксирована, 0.3cm
% ширина второй колонки автоматически рассчитывается из ширины
% страницы (с учётом всевозможных отступов)
\newcommand{\task}[2]{
\begin{tabularx}{\textwidth}{|c|X|}
\cline{1-2}
\makecell*[{{p{0.5cm}}}]{ \centering #1 } &
\makecell*[{{p{\hsize}}}]{ #2 } \\
\cline{1-2}
\end{tabularx}

\vspace{-1pt}

}
% \taskpic[ШИРИНА КАРТИНКИ]{НОМЕР ЗАДАЧИ}{УСЛОВИЕ ЗАДАЧИ}{КАРТИНКА}
% задача с картинкой
% оформлена как таблица с тремя колонками
% первый аргумент - необязательный, по умолчанию ширина картинки равна
% 4cm, но можно выставить свою
% ширина второй колонки (условие задачи) рассчитывается из ширины
% страницы и ширины картинки
\newcommand{\taskpic}[4][4cm]{
\begin{tabularx}{\textwidth}{|c|X|c|}
\cline{1-3}
\makecell*[{{p{0.5cm}}}]{ \centering #2 } &
\makecell*[{{p{\hsize}}}]{ #3 } &
\makecell*[{{p{#1}}}]{ \centering #4} \\
\cline{1-3}
\end{tabularx}

\vspace{-1pt}

}

\parindent=0cm


\pagestyle{empty}
\graphicspath{ {images/} }


\begin{document}
\begin{center}
\begin{Large}
\textsc{ГЦФО. 9 класс. 2014/15.}
\end{Large}
\end{center}
\task{12}{Рядом стоят две пушки, из которых можно стрелять теннисными мячиками под любым углом к горизонту с начальной скоростью $v=20$~м/с. Из пушек одновременно стреляют в бубен, находящийся на расстоянии $L=20$~м по горизонтали, однако удары мячиков о бубен происходят не одновременно. Найдите время между ударами. Расстоянием между пушками, размером бубна, а также сопротивлением воздуха пренебречь. Ускорение свободного падения $g=10$~м/с$^2$.}
\taskpic{13}{Два одинаковых бруса скрепили за середины торцов одинаковыми нерастяжимыми нитями и положили на угол стола (см. рис.). Торцы выступают за края столешницы так, что нити не касаются стола. Коэффициент трения о вертикальную поверхность стола в 3~раза больше, чем о горизонтальную. Известно, что если поставить систему с начальным углом нити к горизонтали $\alpha < 45^\circ$ (см. рис.), то бруски начнут двигаться, тогда как если в начальный момент $\alpha \geqslant 45^\circ$, то система остается неподвижной. Найдите коэффициент трения о горизонтальную поверхность.}{\includegraphics[width=3cm]{13}}
\taskpic{14}{B системе, изображенной на рисунке, пружины имеют жесткости $k_1=100$~Н/м и $k_2=200$ Н/м. К нижнему блоку подвешивают груз массой $M=8$~кг. Система приходит в равновесие. На сколько сместился нижний блок? Пружины, нити и блоки невесомы. Нити нерастяжимы. Ускорение свободного падения $g=10$~м/c$^2$.}{\includegraphics[width=2.5cm]{14}}
\taskpic{15}{На гладкой наклонной плоскости, составляющей с горизонтом угол $\alpha=30^\circ$, расположен массивный клин (см. рис.). На верхней горизонтальной поверхности клина лежит маленькая легкая шайба. Клин отпускают, и он начинает свободно соскальзывать вниз.
\begin{enumerate}
\item Определите величину и направление ускорения движения шайбы относительно наклонной плоскости.
\item Как выглядит движение шайбы в системе отсчета, связанной с клином?
\end{enumerate}
Масса шайбы много меньше массы клина. Трением пренебречь.}{\includegraphics[width=3cm]{15}}
\taskpic{16}{Три одинаковых бревна, имеющих форму цилиндра, сложены так, как показано на рисунке. Какие минимальные коэффициенты трения бревен друг по другу и бревен по земле необходимы для того, чтобы система оставалась в покое?}{\includegraphics[width=3cm]{16}}
\task{17}{ Вася любит принимать ванну и знает, что для него комфортная температура воды 35$^\circ$C. К сожалению, у него на несколько дней отключили холодную воду. Вася померил температуру горячей воды, вытекающей из крана (60$^\circ$C), и заметил, что можно комфортно сидеть в набирающейся ванне, если каждые 7~секунд бросать в нее кубик льда из морозильника. На следующий день оказалось, что ледяные кубики приходится бросать каждые 5~секунд, хотя поток воды из крана такой же. На сколько изменилась температура воды в кране? Тепловыми потерями пренебречь, вода быстро перемешивается и кубики тают быстро.}

\end{document}