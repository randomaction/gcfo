\documentclass[12pt,a4paper]{article}
\usepackage[utf8]{inputenc}
\usepackage{cmap}
\usepackage[OT1,T2A]{fontenc}
\usepackage[russian]{babel}
\usepackage{graphicx}
\usepackage[margin=15mm]{geometry}
\usepackage{amssymb}
\usepackage{tikz}
\usepackage{paralist}

% основные библиотеки
\usepackage{pgfplots, tikz, circuitikz}
\usetikzlibrary{%
arrows,%
calc,%
patterns,%
intersections,%
decorations.pathreplacing,%
decorations.pathmorphing,%
decorations.text,%
decorations.markings,%
shapes,
}
% основные используемые стили
% стиль для стрелки
\tikzset{>=latex,%
every path/.style=thick,%
% платформа: пол или потолок
interface/.style={postaction={draw,decorate,decoration={border,
angle=45, amplitude=0.2cm, segment length=1mm}}},%
% пружина
spring/.style={decorate,decoration={coil,amplitude=1mm, segment
length=1mm},thick},%
% заряд, вершина, просто точка
dot/.style={inner sep=0mm,minimum size=0.1cm,fill,circle},%
% стрелка в середине отрезка
marrow/.style={postaction={draw,decorate,decoration={markings,
mark=at position 0.6 with {\arrow{latex}}}}}}
% обозначение угла
\tikzset{arcnode/.style={
decoration={
markings, raise = 2mm,
mark=at position 0.5 with {
\node[inner sep=0] {#1};
}
},
postaction={decorate}
}
}
% команда для отметки угла: проводит дугу между двумя лучами,
% проведёнными между точками #2--#3 и #2--#4
% первый аргумент - необязательный, стиль линии
% #2,#3,#4 - точки
% #5 - радиус дуги для обозначения угла
% #6 - обозначение угла, например, $\alpha$
\newcommand*\marktheangle[6][]{
\draw[thick,arcnode={#6},#1] let \p2=($(#3)-(#2)$),%
\p3=($(#4)-(#2)$),%
\n2 = {atan2(\x2,\y2)},%
\n3 = {atan2(\x3,\y3)}%
in ($(\n2:#5)+(#2)$) arc (\n2:\n3:#5);
}
% для работы с графиками
\pgfplotsset{compat=newest}
% библиотеки для электричества
\usetikzlibrary{circuits.ee,circuits.ee.IEC}
% амперметр
\tikzset{circuit declare symbol = ammeter}
\tikzset{set ammeter graphic ={draw,generic circle IEC, minimum
size=5mm,info=center:A}}
% вольтметр
\tikzset{circuit declare symbol = voltmeter}
\tikzset{set voltmeter graphic ={draw,generic circle IEC, minimum
size=5mm,info=center:V}}
% кружок
\tikzset{circuit declare symbol = meter}
\tikzset{set meter graphic ={draw,generic circle IEC, minimum
size=5mm}} 
%необходимые пакеты
\usepackage{tabularx}
\usepackage{makecell}
% \task{НОМЕР ЗАДАЧИ}{УСЛОВИЕ ЗАДАЧИ}
% задача без картинки
% оформлена как таблица с двумя колонками
% ширина первой колонки (номер столбца) фиксирована, 0.3cm
% ширина второй колонки автоматически рассчитывается из ширины
% страницы (с учётом всевозможных отступов)
\newcommand{\task}[2]{
\begin{tabularx}{\textwidth}{|c|X|}
\cline{1-2}
\makecell*[{{p{0.5cm}}}]{ \centering #1 } &
\makecell*[{{p{\hsize}}}]{ #2 } \\
\cline{1-2}
\end{tabularx}

\vspace{-1pt}

}
% \taskpic[ШИРИНА КАРТИНКИ]{НОМЕР ЗАДАЧИ}{УСЛОВИЕ ЗАДАЧИ}{КАРТИНКА}
% задача с картинкой
% оформлена как таблица с тремя колонками
% первый аргумент - необязательный, по умолчанию ширина картинки равна
% 4cm, но можно выставить свою
% ширина второй колонки (условие задачи) рассчитывается из ширины
% страницы и ширины картинки
\newcommand{\taskpic}[4][4cm]{
\begin{tabularx}{\textwidth}{|c|X|c|}
\cline{1-3}
\makecell*[{{p{0.5cm}}}]{ \centering #2 } &
\makecell*[{{p{\hsize}}}]{ #3 } &
\makecell*[{{p{#1}}}]{ \centering #4} \\
\cline{1-3}
\end{tabularx}

\vspace{-1pt}

}

\parindent=0cm


\pagestyle{empty}
\graphicspath{ {images/} }


\begin{document}

\begin{center}
\begin{Large}
\textsc{ГЦФО. 9 класс. 2014/15.}
\end{Large}
\end{center}

\small

\taskpic{39}{Легкий жгут жесткости $k$ прикреплен к потолку, а на его конце висят два жука (см. рис.). В таком положении жгут равномерно растянут и его длина от потолка до жуков равна $l$. Потом один жук начинает карабкаться по жгуту вверх с постоянной сокростью $V$ относительно жгута. Как и с какой скоростью относительно потолка будет двигаться второй жук, который продолжает держаться за конец жгута. Считать, что каждый жук хватается за жгут в одной точке. Масса обоих жуков равна $m$, их размерами пренебречь. Ускорение свободного падения равно $g$.}{\includegraphics[width=4cm]{39}}
\taskpic{40}{Два одинаковых проводящих проволочных кольца радиуса $a$ сварили в противоположных точках O и O' как указано на рисунке. Сопротивление единицы длины проволоки равно $\lambda$. Дуги AO и BO равны, их длина $l$. Найти зависимость сопротивления между точками A и B от величины $l$.}{\includegraphics[width=2cm]{40}}
\task{41}{Поршень массы $M$ = 2 кг может с трением скользить внутри вертикальной неподвижной трубы. Сначала поршень прикрепили внутри трубы к потолку пружиной жесткостью $k_1$ = 20 Н/м, длина которой в нерастянутом состоянии $l_1$=60 см. Поршень расположили на уровне середины трубы, отпустили, и он остался неподвижен. Затем опыт повторили, поменяв пружину - жесткость новой пружины стала $k_2$ = 10 Н/м, а длина в нерастянутом состоянии $l_2$ = 20 см. Удивительно, но поршень в середине трубы снова остался неподвижен. При каких знчениях силы трения поршня о трубу это возможно? Влиянием воздуха пренебречь, $g$ = 10 м/с$^2$}
\taskpic{42}{Из куска покрытого изоляцией провода сопротивлением $R$ спаяли кольцо; кольцо свернули в симметричную восьмерку (с одинаковыми петельками). В середине, где провода восьмерки скрещиваются, котакта нет. Точно таким же образом изготовили вторую восьмерку. Источник тока подключают к точкам скрещивания обеих восьмерок (на рис.1 крупно показано подключение одной восьмерки): один из скрещивающихся проводов подключен к "плюсу", а другой - к "минусу". Затем полученные восьмерки спаяли друг с другом в симметричных точках $A$ и $B$(см. рис.2), сопротивление участка провода между $A$ и $B$ равно $R/4$. Каково полное сопротивление этой схемы?}{\includegraphics[width=4cm]{42}}
\taskpic{43}{Велосипед с колесами, имеющими форму равностороннего треугольника, за время $t$ прошел по дороге достаточно большое расстояние $s$. Найдите среднее значение модуля скорости точки, расположенной в вершине колеса. Колеса не проскальзывают по дороге, велосипед не отрывается от земли.}{\includegraphics[width=3cm]{43}}
\taskpic{44}{Экспериментатор взял 4 одинаковых металлических стержня и собрал из них Y-образную фигуру. К концам фигуры экспериментатор присоединил 3 одинаковых больших металлических шара, имеющих температуру $t_1 = 0 ^\circ C$, $t_2 = 50 ^\circ C$ и $t_3 = 100 ^\circ C$ (см. рис.). Экспериментатор обеспечил хороший тепловой контакт стержней с шарами и другими стержнями. Через некоторое время он обнаружил, что первый шар нагрелся на $0,4  ^\circ C$. Какую температуру имели в этот момент два других шара? Считайте, что теплоемкость стержней пренебрежимо мала, а теплообмен с окружающей средой отсутствует. Мощность теплопередачи по стержню пропорциональна разности температур на его концах.}{\includegraphics[width=3cm]{44}}

\end{document}
