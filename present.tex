\documentclass[12pt,a4paper]{article}
\usepackage[utf8]{inputenc}
\usepackage{cmap}
\usepackage[OT1,T2A]{fontenc}
\usepackage[russian]{babel}
\usepackage{graphicx}
\usepackage[margin=15mm]{geometry}
\usepackage{amssymb}
\usepackage{tikz}
\usepackage{paralist}

% основные библиотеки
\usepackage{pgfplots, tikz, circuitikz}
\usetikzlibrary{%
arrows,%
calc,%
patterns,%
intersections,%
decorations.pathreplacing,%
decorations.pathmorphing,%
decorations.text,%
decorations.markings,%
shapes,
}
% основные используемые стили
% стиль для стрелки
\tikzset{>=latex,%
every path/.style=thick,%
% платформа: пол или потолок
interface/.style={postaction={draw,decorate,decoration={border,
angle=45, amplitude=0.2cm, segment length=1mm}}},%
% пружина
spring/.style={decorate,decoration={coil,amplitude=1mm, segment
length=1mm},thick},%
% заряд, вершина, просто точка
dot/.style={inner sep=0mm,minimum size=0.1cm,fill,circle},%
% стрелка в середине отрезка
marrow/.style={postaction={draw,decorate,decoration={markings,
mark=at position 0.6 with {\arrow{latex}}}}}}
% обозначение угла
\tikzset{arcnode/.style={
decoration={
markings, raise = 2mm,
mark=at position 0.5 with {
\node[inner sep=0] {#1};
}
},
postaction={decorate}
}
}
% команда для отметки угла: проводит дугу между двумя лучами,
% проведёнными между точками #2--#3 и #2--#4
% первый аргумент - необязательный, стиль линии
% #2,#3,#4 - точки
% #5 - радиус дуги для обозначения угла
% #6 - обозначение угла, например, $\alpha$
\newcommand*\marktheangle[6][]{
\draw[thick,arcnode={#6},#1] let \p2=($(#3)-(#2)$),%
\p3=($(#4)-(#2)$),%
\n2 = {atan2(\x2,\y2)},%
\n3 = {atan2(\x3,\y3)}%
in ($(\n2:#5)+(#2)$) arc (\n2:\n3:#5);
}
% для работы с графиками
\pgfplotsset{compat=newest}
% библиотеки для электричества
\usetikzlibrary{circuits.ee,circuits.ee.IEC}
% амперметр
\tikzset{circuit declare symbol = ammeter}
\tikzset{set ammeter graphic ={draw,generic circle IEC, minimum
size=5mm,info=center:A}}
% вольтметр
\tikzset{circuit declare symbol = voltmeter}
\tikzset{set voltmeter graphic ={draw,generic circle IEC, minimum
size=5mm,info=center:V}}
% кружок
\tikzset{circuit declare symbol = meter}
\tikzset{set meter graphic ={draw,generic circle IEC, minimum
size=5mm}} 
%необходимые пакеты
\usepackage{tabularx}
\usepackage{makecell}
% \task{НОМЕР ЗАДАЧИ}{УСЛОВИЕ ЗАДАЧИ}
% задача без картинки
% оформлена как таблица с двумя колонками
% ширина первой колонки (номер столбца) фиксирована, 0.3cm
% ширина второй колонки автоматически рассчитывается из ширины
% страницы (с учётом всевозможных отступов)
\newcommand{\task}[2]{
\begin{tabularx}{\textwidth}{|c|X|}
\cline{1-2}
\makecell*[{{p{0.5cm}}}]{ \centering #1 } &
\makecell*[{{p{\hsize}}}]{ #2 } \\
\cline{1-2}
\end{tabularx}

\vspace{-1pt}

}
% \taskpic[ШИРИНА КАРТИНКИ]{НОМЕР ЗАДАЧИ}{УСЛОВИЕ ЗАДАЧИ}{КАРТИНКА}
% задача с картинкой
% оформлена как таблица с тремя колонками
% первый аргумент - необязательный, по умолчанию ширина картинки равна
% 4cm, но можно выставить свою
% ширина второй колонки (условие задачи) рассчитывается из ширины
% страницы и ширины картинки
\newcommand{\taskpic}[4][4cm]{
\begin{tabularx}{\textwidth}{|c|X|c|}
\cline{1-3}
\makecell*[{{p{0.5cm}}}]{ \centering #2 } &
\makecell*[{{p{\hsize}}}]{ #3 } &
\makecell*[{{p{#1}}}]{ \centering #4} \\
\cline{1-3}
\end{tabularx}

\vspace{-1pt}

}

\parindent=0cm


\pagestyle{empty}
\graphicspath{ {images/} }


\begin{document}

\begin{center}
\begin{Large}
\textsc{ГЦФО. 9 класс. 2014/15.}
\end{Large}
\end{center}

\taskpic{15}{На гладкой наклонной плоскости, составляющей с горизонтом угол $\alpha=30^\circ$, расположен массивный клин (см. рис.). На верхней горизонтальной поверхности клина лежит маленькая легкая шайба. Клин отпускают, и он начинает свободно соскальзывать вниз.
\begin{compactenum}
\item Определите величину и направление ускорения движения шайбы относительно наклонной плоскости.
\item Как выглядит движение шайбы в системе отсчета, связанной с клином?
\end{compactenum}
Масса шайбы много меньше массы клина. Трением пренебречь.}{
\begin{tikzpicture}[scale=1.2]
\draw (0,1.5)--(3,0);
\draw[dashed] (0.5,0.5)--(2,0.5);
\filldraw[fill=gray] (0.5,1.27)--(2.5,0.27)--(2.5,1.27)--cycle;
\filldraw[fill=black] (1.6,1.28) rectangle (1.8,1.38);
\draw[thick] (1.5,0.5) arc (180:153.5:0.5) node[midway,left] {$\alpha$};
\end{tikzpicture}
}
\taskpic{16}{Три одинаковых бревна, имеющих форму цилиндра, сложены так, как показано на рисунке. Какие минимальные коэффициенты трения бревен друг по другу и бревен по земле необходимы для того, чтобы система оставалась в покое?}{
\begin{tikzpicture}
\draw[interface] (3,0)--(0,0);
\draw (1,0.5) circle [radius=0.5];
\draw (2,0.5) circle [radius=0.5];
\draw (1.5,1.366) circle [radius=0.5];
\end{tikzpicture}
}
\taskpic[6cm]{18}{На примусе, расходующем $\mu = 0{,}1$~кг бензина в час, стоит котелок, в котором находится $m = 1$~кг воды. График зависимости тепловой мощности $P$, выделяемой в окружающую среду, от времени приведен на рисунке. Постройте график зависимости температуры воды в котелке от времени. Теплоемкость котелка $C = 800$~Дж/$^\circ$C, удельная теплоемкость воды $c_0 = 4200$~Дж/(кг$\cdot^\circ$C). Удельная теплота сгорания бензина $q = 43$~МДж/кг. Начальная температура воды $T=20^\circ$C. Принять, что в любой момент времени температура котелка и воды совпадают.}{
\begin{tikzpicture}
\begin{axis}[
    width=6cm,
    xmax=10.5,
    ymax=1000,
    xtick={0,2,...,10},
    ytick={0,200,...,1000},
    grid=both,
    minor tick num=3,
    major grid style=thick,
    minor grid style=thin,
    axis lines=middle,
    axis line style={->},
    xlabel={$t$, мин},
    ylabel={$P$, Вт},
    /pgf/number format/1000 sep={\,},
]
\addplot[thick,domain=0:10] {1200 * (1 - exp(-x/7))};
\end{axis}
\end{tikzpicture}
}
\taskpic{20}{Любознательный школьник разобрал нагревательный прибор. Оказалось, что схема прибора очень проста (см. рисунок). Школьник вынул все резисторы из схемы и обнаружил, что их сопротивления составляют $R_1=1$~Ом, $R_2=1$~Ом, $R_3=2$~Ом, $R_4=3$~Ом, $R_5=5$~Ом. Но он забыл, какой резистор на каком месте располагается в схеме. Помогите ему собрать прибор по старой схеме таким образом, чтобы его мощность была максимальной. Нагреватель работает от постоянного напряжения.}{
\begin{tikzpicture}[circuit ee IEC,scale=0.75]
\draw (0,0) -- (0.5,0) -- (0.5,0.75) to [resistor=thick] (2,0.75) to [resistor=thick] (3.5,0.75) to [resistor=thick] (5,0.75) -- (5,0) -- (5.5,0);
\draw (0.5,0) -- (0.5,-0.75) to [resistor=thick] (2.75,-0.75) to [resistor=thick] (5,-0.75) -- (5,0);
\end{tikzpicture}
}
\task{21}{Тело роняют над плитой на высоте $h$ от нее. Плита движется вертикально вверх со скоростью $u$. Определите время между двумя последовательными ударами тела о плиту. Удары абсолютно упругие.}
\task{22}{Утюг устроен следующим образом: его нагреватель выключается, если температура утюга становится больше некоторой температуры $t_2$, и включается, как только его температура падает ниже $t_1$ (эти температуры неизвестны). Если включенный утюг стоит с открытой металлической поверхностью, его нагреватель работает в среднем $k=1/4$ всего времени. При этом мощность теплоотдачи можно считать постоянной. Если утюгом начинают гладить, то промежуток времени между последовательными моментами включения нагревателя становится в $n=4/3$ раза меньше. В этом случае мощность теплоотдачи также остается постоянной. Какую часть времени он работает в среднем во втором случае?}
\end{document}
