\documentclass[12pt,a4paper]{article}
\usepackage[utf8]{inputenc}
\usepackage{cmap}
\usepackage[OT1,T2A]{fontenc}
\usepackage[russian]{babel}
\usepackage{graphicx}
\usepackage[margin=15mm]{geometry}
\usepackage{amssymb}
\usepackage{tikz}
\usepackage{paralist}

% основные библиотеки
\usepackage{pgfplots, tikz, circuitikz}
\usetikzlibrary{%
arrows,%
calc,%
patterns,%
intersections,%
decorations.pathreplacing,%
decorations.pathmorphing,%
decorations.text,%
decorations.markings,%
shapes,
}
% основные используемые стили
% стиль для стрелки
\tikzset{>=latex,%
every path/.style=thick,%
% платформа: пол или потолок
interface/.style={postaction={draw,decorate,decoration={border,
angle=45, amplitude=0.2cm, segment length=1mm}}},%
% пружина
spring/.style={decorate,decoration={coil,amplitude=1mm, segment
length=1mm},thick},%
% заряд, вершина, просто точка
dot/.style={inner sep=0mm,minimum size=0.1cm,fill,circle},%
% стрелка в середине отрезка
marrow/.style={postaction={draw,decorate,decoration={markings,
mark=at position 0.6 with {\arrow{latex}}}}}}
% обозначение угла
\tikzset{arcnode/.style={
decoration={
markings, raise = 2mm,
mark=at position 0.5 with {
\node[inner sep=0] {#1};
}
},
postaction={decorate}
}
}
% команда для отметки угла: проводит дугу между двумя лучами,
% проведёнными между точками #2--#3 и #2--#4
% первый аргумент - необязательный, стиль линии
% #2,#3,#4 - точки
% #5 - радиус дуги для обозначения угла
% #6 - обозначение угла, например, $\alpha$
\newcommand*\marktheangle[6][]{
\draw[thick,arcnode={#6},#1] let \p2=($(#3)-(#2)$),%
\p3=($(#4)-(#2)$),%
\n2 = {atan2(\x2,\y2)},%
\n3 = {atan2(\x3,\y3)}%
in ($(\n2:#5)+(#2)$) arc (\n2:\n3:#5);
}
% для работы с графиками
\pgfplotsset{compat=newest}
% библиотеки для электричества
\usetikzlibrary{circuits.ee,circuits.ee.IEC}
% амперметр
\tikzset{circuit declare symbol = ammeter}
\tikzset{set ammeter graphic ={draw,generic circle IEC, minimum
size=5mm,info=center:A}}
% вольтметр
\tikzset{circuit declare symbol = voltmeter}
\tikzset{set voltmeter graphic ={draw,generic circle IEC, minimum
size=5mm,info=center:V}}
% кружок
\tikzset{circuit declare symbol = meter}
\tikzset{set meter graphic ={draw,generic circle IEC, minimum
size=5mm}} 
%необходимые пакеты
\usepackage{tabularx}
\usepackage{makecell}
% \task{НОМЕР ЗАДАЧИ}{УСЛОВИЕ ЗАДАЧИ}
% задача без картинки
% оформлена как таблица с двумя колонками
% ширина первой колонки (номер столбца) фиксирована, 0.3cm
% ширина второй колонки автоматически рассчитывается из ширины
% страницы (с учётом всевозможных отступов)
\newcommand{\task}[2]{
\begin{tabularx}{\textwidth}{|c|X|}
\cline{1-2}
\makecell*[{{p{0.5cm}}}]{ \centering #1 } &
\makecell*[{{p{\hsize}}}]{ #2 } \\
\cline{1-2}
\end{tabularx}

\vspace{-1pt}

}
% \taskpic[ШИРИНА КАРТИНКИ]{НОМЕР ЗАДАЧИ}{УСЛОВИЕ ЗАДАЧИ}{КАРТИНКА}
% задача с картинкой
% оформлена как таблица с тремя колонками
% первый аргумент - необязательный, по умолчанию ширина картинки равна
% 4cm, но можно выставить свою
% ширина второй колонки (условие задачи) рассчитывается из ширины
% страницы и ширины картинки
\newcommand{\taskpic}[4][4cm]{
\begin{tabularx}{\textwidth}{|c|X|c|}
\cline{1-3}
\makecell*[{{p{0.5cm}}}]{ \centering #2 } &
\makecell*[{{p{\hsize}}}]{ #3 } &
\makecell*[{{p{#1}}}]{ \centering #4} \\
\cline{1-3}
\end{tabularx}

\vspace{-1pt}

}

\parindent=0cm


\pagestyle{empty}
\graphicspath{ {images/} }


\begin{document}

\begin{center}
\begin{Large}
\textsc{ГЦФО. 9 класс. 2014/15.}
\end{Large}
\end{center}

\task{29}{На гладком горизонтальном столе лежат два одинаковых бруска, соединенных пружиной жесткости $k$ и длины $l_0$. На левый брусок внезапно начинает действовать постоянная сила $F$, направленная вдоль пружины. Найдите минимальное и максимальное расстояние между брусками.}
\task{31}{Локомотив с постоянной силой тяги $F$ начал двигаться к стоящему вагону и столкнулся с ним через время $\Delta t$. Найдите время между последующими соударениями локомотива с этим вагоном. Удар упругий. Трением в осях колес пренебречь. Массы вагона и локомотива не одинаковы.}
\taskpic{34}{На концах длинной нити подвешены грузы массы $m$ каждый. Нить перекинута через два легких маленьких блока, расположенных на расстоянии $2l$ друг от друга. К ней посередине между блоками прикрепляют груз массы $2m$, и система приходит в движение. Найдите скорость грузов по истечении достаточно большого промежутка времени.}{\includegraphics[width=4cm]{34}}
\taskpic{35}{Деревянная и металлическая однородные балки соединены, как показано на рисунке. Размеры, указанные на рисунке, составляют $a=10$ см, $b=5$ см, $c=35$ см. Темным цветом изображена металлическая балка. Известно, что вся конструкция может плавать, полностью погрузившись в воду. Какой угол при этом составляет длинная балка с вертикалью?}{\includegraphics[width=4cm]{35}}
\taskpic{36}{Из однородной проволоки спаяли схему, состоящую из колец (см. рис.). Внешнее кольцо имеет диаметр $D$, внутрь него вложены два кольца вдвое меньшего диаметра; в каждое из меньших колец вложены еще два, которые меньше еще вдвое. В местах касания колец есть электрический контакт. Клемма Ф присоединена к середине дуги правой полуокружности. Найдите сопротивление этой схемы между клеммами Д и Ф. Каким будет сопротивление схемы, в которой кольца из проволоки вкладываются по данному правилу до бесконечности? Сопротивление единицы длины проволоки $\lambda$.}{\includegraphics[width=4cm]{36}}
\taskpic{37}{Труба, сечение которой является квадратом со стороной $a = 20$ см, закрыта поршнем. К трубе присоединена вертикальная трубка. Часть трубы, находящаяся справа от поршня, полностью заполнена водой. Чтобы удерживать поршень в равновесии, к нему необходимо прикладывать силу $F = 16$ Н, направленную вправо. Каков уровень воды в трубке? Плотность воды $\rho = 1000$ кг/м$^3$, ускорение свободного падения $g = 10$ м/с$^2$. Трение отсутствует.}{\includegraphics[width=4cm]{37}}

\end{document}
